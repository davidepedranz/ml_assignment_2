\section{Introduction}
\label{sec:introduction}

Machine Learning techniques allow to extract knowledge from training data and use it to classify or predict new examples.
There are different learning algorithms for different problems.
In this assignment, we will focus on the classification problem.
In particular, we will run the following algorithms and compare their performances:
\begin{itemize}
    \item Naive Bayes,
    \item Support Vector Machine,
    \item Random Forest.
\end{itemize}

Most algorithm require to set some parameters in advance, before the learning phase.
These parameters are called hyperparameters.
It is usually tricky to choose this parameters in advance, so a common way to overcome this problem is to use a part of the training data to tune their values.
We used the process called k-fold cross validation, which is described in \cref{sec:k_fold_cross_validation}.

We use the Python library \texttt{Scikit-Learn}\footnote{\url{http://scikit-learn.org/}}.
Scikit-Learn implement most of the commonly used Machine Learning algorithms and provides utilities to generate learning problems, split the data and measure the performances.
